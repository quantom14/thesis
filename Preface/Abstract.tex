\newgeometry{
left=4cm,
right=2.5cm,
top=2cm,
bottom=4cm
}
\chapter*{Abstract}
\addcontentsline{toc}{section}{Abstract}

Measurement-induced phase transitions arise from the interplay of coherent quantum dynamics and local measurements. For example, coherent dynamics can generate entanglement and long-ranged correlations, while local measurements destroy them. By tuning the strength of the measurements, the system undergoes a phase transition at the level of individual stochastic measurement trajectories. It is only witnessed by non-linear functions in the density operator, and due to the stochastic nature of the measurement outcomes, the average density operator masks the transition. 

In this thesis, we explore how this type of transition arises in bosonic and fermionic models that are subject to dephasing or measurement and consider ways to detect the transition in experiments. We first give an overview of the transition and describe the difficulties in identifying the exact location of the transition and finding experimental protocols to detect them. We then show that in a homodyne detection setup, it is impossible to reconstruct the nonlinear correlation function that witnesses the transition, using only the linear information from homodyne currents, and we discuss the difficulties that arise when considering the experimental detection of this transition. We finally show that features of the competition between coherent and dissipative dynamics are already present at short times during the system evolution. We also present a protocol that displays the characteristic behavior of the transition, starting from an infinite temperature state. 

In all projects, we employ a range of numerical techniques, such as Monte-Carlo wavefunction methods, which allow us to simulate the open quantum systems dynamics and enable us to explore the presented models.

\cleardoublepage
\restoregeometry

