\chapter{Conclusion}
\thispagestyle{empty}
\label{chap:conclusion}

In this thesis, we have explored measurement-induced phase transitions in continuous time models that result from the competition between coherent and dissipative dynamics. These transitions manifest in non-linear steady-state properties in many-particle quantum systems accessible only at the trajectory level since the system approaches the structureless infinite temperature state at long times, independent of the measurement or dissipation strength. To experimentally measure non-linear quantities, one would require accessing individual trajectories multiple times, making it challenging to find suitable quantities that reveal the phase transition in an experiment. We now summarize the ways in which we explored ways to understand these transitions better and find a way to detect them experimentally. 

In chapter~\ref{chap:MIPT_bosons}, we presented the measurement-induced phase transition that arises from the interplay between coherent time evolution and dissipative dynamics in a bosonic $1$D chain. The von Neumann entropy best characterizes the transition; for small dissipation strengths, we observed a volume-law phase where the entropy increases linearly with subsystem size, while for large dissipation strengths, we observed an area-law phase, where the entropy is approximately constant. At intermediate dissipation strengths, the entropy scales approximately logarithmically. To pinpoint the exact location of the transition, we performed a scaling collapse analysis of our numerically simulated data and concluded that we could not pinpoint the transition accurately. This was largely due to the fact that we were not able to simulate large enough system sizes, which in random circuit models, where such transitions were studied first, is not a problem as the numerical tools allow simulation of much larger system sizes. We considered two dissipative processes: dephasing and single-particle gain or loss. We showed that the exact type of dissipation did not make a difference. The $U(1)$ symmetry is broken for single particle loss or gain; however, the outcome of the scaling collapse analysis did not change. 

Then, in chapter~\ref{chap:MIPT_continuous_measurement} we shifted our focus to experimentally detecting the phase transition we studied in chapter~\ref{chap:MIPT_bosons}. We considered free fermions in a $1$D chain subject to weak measurements, which can be simulated for larger system sizes under the evolution of a quadratic Hamiltonian. The method for experimental detection we proposed was based on first rewriting a non-linear correlation function in the density operator as a function of local number operators. The transition is characterized by this correlation function, which decays algebraically in the small measurement regime and exponentially in the large measurement regime. Secondly, in the context of weak measurements, the natural choice was to consider a homodyne detection setup, where the measured output signals are homodyne currents, which contain only linear information in the density operator. In our case, this means that by averaging many homodyne currents, we can measure the local densities and use this to reconstruct the correlation function that characterizes the transition. However, we proved that extracting the correlation function using only homodyne currents is impossible, as the non-linear cross-product terms cancel out when computing the averages, rendering the protocol not feasible for experimental detection.

Finally, in chapter~\ref{chap:short_time_dynamics}, we asked whether detecting the competition between coherent and dissipative dynamics at early times during the system evolution is possible, considering the same model as in chapter\ref{chap:MIPT_bosons}. First, we showed that the von Neumann entropy at short times displays characteristic behavior associated with the phase transition. Secondly, coherent and dissipative dynamics are competing at short times, and we used a linear function, namely the population imbalance, to analyze the early time dynamics. We fitted an exponentially decaying Bessel function to the simulated data and analyzed the fitting parameters. We observed that the damping rate increases to approximately the dissipation strength at which we expect the transition to occur and then decreases algebraically. Lastly, we proposed a protocol in which randomly rotated states are sampled from the infinite temperature state, as they provide a way to reproduce initial states reliably. By measuring the number operator in a specific way, we showed that we could construct a correlation function that is able to display similar characteristic behavior we saw in the analysis of the population imbalance, therefore providing an experimentally feasible method to visualize features of the transition even if we cannot probe the transition itself. 

To summarize, we have analyzed a range of continuous time models in which measurement-induced phase transitions appear and explored some ways in which the competition between coherent and dissipative dynamics can be observed experimentally. Although these transitions can be studied using standard numerical methods to simulate the models, finding an experimentally feasible protocol is difficult, as we showed in this thesis. The main difficulty comes from the fact that the transition is masked at the level of the density operator and the restriction of being unable to access individual trajectories multiple times. 